\documentclass[12pt,a4paper]{article}

% Essential packages
\usepackage{mathpazo}
\usepackage[T1]{fontenc}
\usepackage[utf8]{inputenc}
\usepackage{geometry}
\usepackage{xcolor, soul}
\usepackage{graphicx}
\usepackage{enumitem}
\usepackage{amsmath,amsthm}
\usepackage{amssymb}
\usepackage{fancyhdr}
\usepackage{lastpage}
\usepackage{float}
\usepackage{tikz}
\usepackage{fontawesome}

% Page geometry
\geometry{
    a4paper,
    top=2.5cm,
    bottom=2.5cm,
    left=2.5cm,
    right=2.5cm,
    headheight=15pt
}

% Custom colors
\definecolor{dekutPurple}{RGB}{128, 0, 128}
\definecolor{UM_Brown}{HTML}{3D190D}

% Header and footer style
\pagestyle{fancy}
\fancyhf{}
\renewcommand{\headrulewidth}{0pt}
\renewcommand{\footrulewidth}{0.5pt}
\fancyfoot[C]{Page \thepage\ of \pageref{LastPage}}

\begin{document}

% Title section
\textcolor{UM_Brown}{
\begin{minipage}{1.0\textwidth}
\centering{\includegraphics[width=7cm]{figures/dekut.jpg}}
    \begin{center}
        \textbf{\large \textsc{Dedan Kimathi University of Technology} }\\
        \textbf{\textbf{2025/2026 University Examination} }\\
        \vspace{5pt}
        \textbf{\large \textsc{Second Semester of Fourth Year,} }\\
        \textbf{\large \textsc{Bachelor of Science in Mechatronic Engineering} }\\
        \vspace{10pt}
        \textbf{\large \textsc{EMT 4201: Spatial Mechanics} }\\
        \vspace{25pt}
    \end{center}
\end{minipage}
\vspace{5pt}
\textbf{DATE: \qquad Jan. 2026} \hfill \textbf{DURATION: } \faClockO\ \textbf{2 hours}
\hrule
\vspace{5pt}
% \textbf{Instructions: \footnote{\faPencil\ \textit{orcid.org/0000-0002-7326-7502} } }
\begin{itemize}
    \item [\fbox{\textbf{1.}}] This examination contains \textbf{70 marks} in total.
    \item [\fbox{\textbf{2.}}] \textbf{Question 1} is \textbf{compulsory}.
    \item [\fbox{\textbf{3.}}] Attempt \textbf{one} question in \textbf{SECTION A} and \textbf{one} question in \textbf{SECTION B}.
\end{itemize}
\vspace{5pt}
\hrule
\vspace{5pt}
}
\hrule height 0.1pt
\vspace{-14pt}
\subsection*{Question 1: Compulsory \hfill \textbf{[30 marks]}}
\vspace{-5pt}
\hrule height 0.1pt
\vspace{10pt}

\begin{enumerate}[label=(\alph*)]
\item Define and distinguish between the following joint types:
\begin{enumerate}[label=(\roman*)]
\item Revolute joint and prismatic joint \hfill \textbf{[2 marks]}
\item Spherical joint and universal joint \hfill \textbf{[2 marks]}
\item Helical joint and cylindrical joint \hfill \textbf{[2 marks]}
\end{enumerate}

\item State the Gr\"{u}bler's equation for spatial mechanisms and explain each term. \hfill \textbf{[3 marks]}

\item A 4-finger underactuated gripper for fruit harvesting has 12 revolute joints (3 per finger) and 13 links including the palm. Calculate the total degrees of freedom without coupling constraints, then determine how many coupling constraints are needed to achieve single-motor control. \hfill \textbf{[4 marks]}

\item Explain the concept of Remote Center of Motion (RCM) constraint in minimally invasive surgery and how it affects the effective degrees of freedom of a surgical robot. \hfill \textbf{[3 marks]}

\item A point $P(5, 3)$ undergoes a rotation of $40^\circ$ about the origin followed by a translation of $(3, -2)$. Using complex number representation, determine the final position of point P. \hfill \textbf{[4 marks]}

\item For a SCARA robot with link lengths $L_1 = 350$ mm and $L_2 = 250$ mm, write the forward kinematics equations to find the end-effector position $(x, y)$ given joint angles $\theta_1$ and $\theta_2$. \hfill \textbf{[4 marks]}

\item A CNC router needs to cut an elliptical path with semi-major axis $a = 80$ mm and semi-minor axis $b = 50$ mm, rotated by $25^\circ$ and centered at $(150, 100)$ mm. Write the parametric equations for this tool path. \hfill \textbf{[3 marks]}

\item Explain the phenomenon of gimbal lock in Euler angle representations and state at which angles it occurs for the ZYX convention. \hfill \textbf{[3 marks]}
\end{enumerate}

\newpage

\hrule height 0.1pt
\vspace{-14pt}
\subsection*{SECTION A: Attempt \fbox{Only One} Question (20 marks)}
\vspace{-5pt}
\hrule height 0.1pt
\vspace{10pt}

\subsection*{Question 2: Vision-Guided Robotic System \hfill \textbf{[20 marks]}}

A vision-guided robotic system has the following configuration:

\begin{enumerate}[label=(\alph*)]
\item The camera system observes both a workpiece and the robot base. The homogeneous transformation matrices are given as:
\begin{equation*}
H_1 =
\begin{bmatrix}
0 & -1 & 0 & 3\\
1 & 0 & 0 & 5\\
0 & 0 & -1 & 4\\
0 & 0 & 0 & 1
\end{bmatrix},
\quad
H_2 =
\begin{bmatrix}
1 & 0 & 0 & -5\\
0 & -1 & 0 & 10\\
0 & 0 & -1 & 8\\
0 & 0 & 0 & 1
\end{bmatrix}
\end{equation*}
where $H_1$ represents the workpiece position relative to the camera, and $H_2$ represents the robot base position relative to the camera.
\begin{enumerate}[label=(\roman*)]
\item Find the inverse of $H_2$ and verify it is correct by multiplication \hfill \textbf{[4 marks]}
\item Calculate the position and orientation of the workpiece relative to the robot base frame \hfill \textbf{[4 marks]}
\item If the camera is rotated $45^\circ$ about its current Y-axis, determine the new transformation matrix for the camera and calculate how this affects the workpiece position \hfill \textbf{[6 marks]}
\end{enumerate}

\item A point $P(3,1,2)$ is attached to a mobile frame that undergoes:
\begin{itemize}
\item A rotation of $180^\circ$ about the Z-axis
\item Followed by a translation of $(0,2,1)$
\end{itemize}
\begin{enumerate}[label=(\roman*)]
\item Form the homogeneous transformation matrix for this motion \hfill \textbf{[3 marks]}
\item Calculate the final coordinates of point $P$ in the base frame \hfill \textbf{[3 marks]}
\end{enumerate}
\end{enumerate}

\subsection*{Question 3: 3-DOF Robot Manipulator \hfill \textbf{[20 marks]}}

A 3-DOF robot manipulator performs a pick-and-place operation. The manipulator undergoes three consecutive rotations followed by a translation. Given that:
\begin{itemize}
\item Initial rotation: $45^\circ$ about the base Z-axis
\item Second rotation: $60^\circ$ about the resulting Y-axis
\item Third rotation: $30^\circ$ about the resulting X-axis
\item Translation: $(2,3,1)^\top$ units in the final coordinate frame
\end{itemize}

\begin{enumerate}[label=(\alph*)]
\item For each individual rotation:
\begin{enumerate}[label=(\roman*)]
\item Derive the rotation matrices for all three rotations, showing all trigonometric substitutions \hfill \textbf{[6 marks]}
\item Verify that each rotation matrix is orthogonal by checking $RR^\top = I$ \hfill \textbf{[3 marks]}
\end{enumerate}

\item Determine the composite transformation:
\begin{enumerate}[label=(\roman*)]
\item Calculate the composite rotation matrix by multiplying the individual matrices in the correct sequence \hfill \textbf{[4 marks]}
\item Form the homogeneous transformation matrix incorporating the translation \hfill \textbf{[2 marks]}
\end{enumerate}

\item Using the sensor point $\mathbf{P}(1,1,0)^\top$:
\begin{enumerate}[label=(\roman*)]
\item Calculate its final position after all transformations \hfill \textbf{[3 marks]}
\item Verify that the distance from the origin to point P remains constant after pure rotation \hfill \textbf{[2 marks]}
\end{enumerate}
\end{enumerate}

\newpage

\hrule height 0.1pt
\vspace{-14pt}
\subsection*{SECTION B: Attempt \fbox{Only One} Question (20 marks)}
\vspace{-5pt}
\hrule height 0.1pt
\vspace{10pt}

\subsection*{Question 4: Pipe Welding Robot Orientation \hfill \textbf{[20 marks]}}

A robotic welding torch must maintain perpendicular orientation to a horizontal pipe surface (aligned with X-axis) of radius $R = 100$ mm at four cardinal positions.

\begin{enumerate}[label=(\alph*)]
\item For each welding position, determine the required surface normal vector:
\begin{enumerate}[label=(\roman*)]
\item Top position ($\theta = 0^\circ$) \hfill \textbf{[1 mark]}
\item Right side ($\theta = 90^\circ$) \hfill \textbf{[1 mark]}
\item Bottom position ($\theta = 180^\circ$) \hfill \textbf{[1 mark]}
\item Left side ($\theta = 270^\circ$) \hfill \textbf{[1 mark]}
\end{enumerate}

\item Derive the rotation matrix for the top position ($\theta = 0^\circ$) where the torch Z-axis must align with the surface normal $(0, 1, 0)$. \hfill \textbf{[4 marks]}

\item Calculate the rotation matrix for the bottom position ($\theta = 180^\circ$) and verify it is orthogonal by checking $RR^\top = I$. \hfill \textbf{[5 marks]}

\item Extract the Euler angles (Roll, Pitch, Yaw) using ZYX convention for:
\begin{enumerate}[label=(\roman*)]
\item Top position rotation matrix \hfill \textbf{[2 marks]}
\item Bottom position rotation matrix \hfill \textbf{[2 marks]}
\end{enumerate}

\item Explain why all four cardinal positions involve only X-axis rotations and discuss whether gimbal lock occurs at any position. \hfill \textbf{[3 marks]}
\end{enumerate}

\subsection*{Question 5: Arbitrary Axis Rotation using Rodrigues' Formula \hfill \textbf{[20 marks]}}

A component needs to be rotated by $90^\circ$ about an arbitrary axis $\mathbf{V} = (2, 2, 2)^\top$ using Rodrigues' formula: $R = I + \sin\theta\, W + (1-\cos\theta)\, W^2$

\begin{enumerate}[label=(\alph*)]
\item Normalize the rotation axis vector $\mathbf{V}$:
\begin{enumerate}[label=(\roman*)]
\item Calculate the magnitude $\|\mathbf{V}\|$ \hfill \textbf{[1 mark]}
\item Determine the unit vector components $V_x, V_y, V_z$ \hfill \textbf{[2 marks]}
\end{enumerate}

\item Construct the skew-symmetric matrix $W$ from the unit vector. \hfill \textbf{[3 marks]}

\item Calculate $W^2$ by multiplying $W \times W$. \hfill \textbf{[4 marks]}

\item Apply Rodrigues' formula to obtain the final rotation matrix:
\begin{enumerate}[label=(\roman*)]
\item Evaluate $\sin(90^\circ)$ and $(1-\cos(90^\circ))$ \hfill \textbf{[1 mark]}
\item Compute $R = I + W + W^2$ \hfill \textbf{[3 marks]}
\end{enumerate}

\item Apply the rotation matrix to transform the point $P(1, 0, 0)^\top$ and determine its final position. \hfill \textbf{[3 marks]}

\item Verify that the rotation matrix is orthogonal by checking $\det(R) = +1$. \hfill \textbf{[3 marks]}
\end{enumerate}

\vfill
\begin{flushright}
\textit{End of Questions}
\end{flushright}

\end{document}
