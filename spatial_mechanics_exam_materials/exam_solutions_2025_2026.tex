\documentclass[12pt,a4paper]{article}

% Essential packages
\usepackage{mathpazo}
\usepackage[T1]{fontenc}
\usepackage[utf8]{inputenc}
\usepackage{geometry}
\usepackage{xcolor, soul}
\usepackage{graphicx}
\usepackage{enumitem}
\usepackage{amsmath,amsthm}
\usepackage{amssymb}
\usepackage{fancyhdr}
\usepackage{lastpage}
\usepackage{float}
\usepackage{tikz}
\usepackage{fontawesome}

% Page geometry
\geometry{
    a4paper,
    top=2.5cm,
    bottom=2.5cm,
    left=2.5cm,
    right=2.5cm,
    headheight=15pt
}

% Custom colors
\definecolor{dekutPurple}{RGB}{128, 0, 128}
\definecolor{UM_Brown}{HTML}{3D190D}

% Header and footer style
\pagestyle{fancy}
\fancyhf{}
\renewcommand{\headrulewidth}{0pt}
\renewcommand{\footrulewidth}{0.5pt}
\fancyfoot[C]{Page \thepage\ of \pageref{LastPage}}

\begin{document}

% Title section
\textcolor{UM_Brown}{
\begin{minipage}{1.0\textwidth}
\centering{\includegraphics[width=7cm]{figures/dekut.jpg}}
    \begin{center}
        \textbf{\large \textsc{Dedan Kimathi University of Technology} }\\
        \textbf{\textbf{2025/2026 University Examination - Solutions} }\\
        \vspace{5pt}
        \textbf{\large \textsc{Second Semester of Fourth Year,} }\\
        \textbf{\large \textsc{Bachelor of Science in Mechatronic Engineering} }\\
        \vspace{10pt}
        \textbf{\large \textsc{EMT 4201: Spatial Mechanics} }\\
        \vspace{25pt}
    \end{center}
\end{minipage}
\vspace{5pt}
\textbf{DATE: \qquad Jan. 2026} \hfill \textbf{DURATION: } \faClockO\ \textbf{2 hours}
\hrule
\vspace{5pt}
\begin{itemize}
    \item [\fbox{\textbf{1.}}] This examination contains \textbf{70 marks} in total.
    \item [\fbox{\textbf{2.}}] \textbf{Question 1} is \textbf{compulsory}.
    \item [\fbox{\textbf{3.}}] Attempt \textbf{one} question in \textbf{SECTION A} and \textbf{one} question in \textbf{SECTION B}.
\end{itemize}
\vspace{5pt}
\hrule
\vspace{5pt}
}
\hrule height 0.1pt
\vspace{-14pt}
\subsection*{Question 1: Compulsory \hfill \textbf{[30 marks]}}
\vspace{-5pt}
\hrule height 0.1pt
\vspace{10pt}

\begin{enumerate}[label=(\alph*)]
\item Define and distinguish between the following joint types:

\begin{enumerate}[label=(\roman*)]
\item \textbf{Revolute joint and prismatic joint} \hfill \textbf{[2 marks]}

\textbf{Revolute joint:} Allows rotational motion about a fixed axis, providing 1 DoF. {\color{red}\checkmark}

\textbf{Prismatic joint:} Allows linear sliding motion along a fixed axis, providing 1 DoF. {\color{red}\checkmark}

\item \textbf{Spherical joint and universal joint} \hfill \textbf{[2 marks]}

\textbf{Spherical joint:} Ball-and-socket joint allowing rotation about 3 perpendicular axes, providing 3 DoF. {\color{red}\checkmark}

\textbf{Universal joint:} Two revolute joints with perpendicular axes, providing 2 DoF. {\color{red}\checkmark}

\item \textbf{Helical joint and cylindrical joint} \hfill \textbf{[2 marks]}

\textbf{Helical joint:} Coupled rotation and translation (like a screw), providing 1 DoF. {\color{red}\checkmark}

\textbf{Cylindrical joint:} Independent rotation and translation along same axis, providing 2 DoF. {\color{red}\checkmark}
\end{enumerate}

\item State the Gr\"{u}bler's equation for spatial mechanisms and explain each term. \hfill \textbf{[3 marks]}

\textbf{Gr\"{u}bler's equation:} $F = 6(n-1) - \sum_{i=1}^{j} c_i$ {\color{red}\checkmark}

Where:
\begin{itemize}
\item $F$ = degrees of freedom of the mechanism {\color{red}\checkmark}
\item $n$ = number of links (including ground)
\item $c_i$ = number of constraints imposed by joint $i$
\item The factor 6 represents the maximum DoF of a rigid body in 3D space {\color{red}\checkmark}
\end{itemize}

\item A 4-finger underactuated gripper for fruit harvesting has 12 revolute joints (3 per finger) and 13 links including the palm. Calculate the total degrees of freedom without coupling constraints, then determine how many coupling constraints are needed to achieve single-motor control. \hfill \textbf{[4 marks]}

\textbf{Without coupling:}
$$F = 6(13-1) - 12 \times 5 = 6(12) - 60 = 72 - 60 = 12 \text{ DoF}$$ {\color{red}\checkmark}

\textbf{For single-motor control:} Need $F = 1$ DoF {\color{red}\checkmark}

Number of coupling constraints needed:
$$C_{coupling} = 12 - 1 = 11 \text{ constraints}$$ {\color{red}\checkmark}

This allows 1 motor to control all 12 joints through mechanical coupling. {\color{red}\checkmark}

\item Explain the concept of Remote Center of Motion (RCM) constraint in minimally invasive surgery and how it affects the effective degrees of freedom of a surgical robot. \hfill \textbf{[3 marks]}

\textbf{RCM constraint:} A kinematic constraint that forces the instrument shaft to pivot about a fixed point (the trocar insertion point) on the patient's body wall. {\color{red}\checkmark}

\textbf{Effect on DoF:} The trocar removes 2 translational degrees of freedom (instrument cannot translate perpendicular to insertion axis). {\color{red}\checkmark}

A surgical robot with 5 internal DoF has only 3 effective DoF inside the patient after RCM constraint is applied. {\color{red}\checkmark}

\item A point $P(5, 3)$ undergoes a rotation of $40^\circ$ about the origin followed by a translation of $(3, -2)$. Using complex number representation, determine the final position of point P. \hfill \textbf{[4 marks]}

\textbf{Initial point:} $z_0 = 5 + 3i$ {\color{red}\checkmark}

\textbf{Rotation by $40^\circ$:}
$$z_1 = z_0 \cdot e^{i40^\circ} = (5 + 3i)(\cos 40^\circ + i \sin 40^\circ)$$
$$z_1 = (5 + 3i)(0.766 + 0.643i) = 3.83 - 0.899i + 2.30i + 1.93i^2$$
$$z_1 = 1.90 + 1.40i$$ {\color{red}\checkmark}

\textbf{Translation by $(3, -2)$:}
$$z_{final} = z_1 + (3 - 2i) = (1.90 + 1.40i) + (3 - 2i)$$
$$z_{final} = 4.90 - 0.60i$$ {\color{red}\checkmark}

\textbf{Final position:} $(4.90, -0.60)$ {\color{red}\checkmark}

\item For a SCARA robot with link lengths $L_1 = 350$ mm and $L_2 = 250$ mm, write the forward kinematics equations to find the end-effector position $(x, y)$ given joint angles $\theta_1$ and $\theta_2$. \hfill \textbf{[4 marks]}

\textbf{Using complex numbers:}
$$z_{end} = L_1 e^{i\theta_1} + L_2 e^{i(\theta_1 + \theta_2)}$$ {\color{red}\checkmark}

\textbf{Cartesian form:}
$$x = L_1 \cos\theta_1 + L_2 \cos(\theta_1 + \theta_2)$$ {\color{red}\checkmark}
$$y = L_1 \sin\theta_1 + L_2 \sin(\theta_1 + \theta_2)$$ {\color{red}\checkmark}

\textbf{With values:}
$$x = 350\cos\theta_1 + 250\cos(\theta_1 + \theta_2)$$
$$y = 350\sin\theta_1 + 250\sin(\theta_1 + \theta_2)$$ {\color{red}\checkmark}

\item A CNC router needs to cut an elliptical path with semi-major axis $a = 80$ mm and semi-minor axis $b = 50$ mm, rotated by $25^\circ$ and centered at $(150, 100)$ mm. Write the parametric equations for this tool path. \hfill \textbf{[3 marks]}

\textbf{Unrotated ellipse:} $z(t) = 80\cos t + 50i\sin t$ {\color{red}\checkmark}

\textbf{Rotation by $25^\circ$:}
$$x(t) = 150 + 80\cos t \cos 25^\circ - 50\sin t \sin 25^\circ$$
$$y(t) = 100 + 80\cos t \sin 25^\circ + 50\sin t \cos 25^\circ$$ {\color{red}\checkmark}

\textbf{Simplified:}
$$x(t) = 150 + 72.50\cos t - 21.13\sin t$$
$$y(t) = 100 + 33.81\cos t + 45.32\sin t$$ {\color{red}\checkmark}

\item Explain the phenomenon of gimbal lock in Euler angle representations and state at which angles it occurs for the ZYX convention. \hfill \textbf{[3 marks]}

\textbf{Gimbal lock:} A singularity where two rotation axes align, causing loss of one degree of freedom in orientation representation. {\color{red}\checkmark}

\textbf{For ZYX convention:} Occurs when pitch angle $\beta = \pm 90^\circ$ {\color{red}\checkmark}

At this configuration, roll and yaw rotations become indistinguishable, preventing unique Euler angle extraction. {\color{red}\checkmark}
\end{enumerate}

\newpage

\hrule height 0.1pt
\vspace{-14pt}
\subsection*{SECTION A: Attempt \fbox{Only One} Question (20 marks)}
\vspace{-5pt}
\hrule height 0.1pt
\vspace{10pt}

\subsection*{Question 2: Vision-Guided Robotic System - Solution \hfill \textbf{[20 marks]}}

\begin{enumerate}[label=(\alph*)]
\item The camera system observes both a workpiece and the robot base.

\begin{enumerate}[label=(\roman*)]
\item \textbf{Find the inverse of $H_2$ and verify by multiplication} \hfill \textbf{[4 marks]}

For homogeneous transformation $H = \begin{bmatrix} R & \mathbf{d} \\ 0 & 1 \end{bmatrix}$, inverse is $H^{-1} = \begin{bmatrix} R^\top & -R^\top\mathbf{d} \\ 0 & 1 \end{bmatrix}$ {\color{red}\checkmark}

From $H_2$: $R_2 = \begin{bmatrix} 1 & 0 & 0 \\ 0 & -1 & 0 \\ 0 & 0 & -1 \end{bmatrix}$, $\mathbf{d}_2 = \begin{bmatrix} -5 \\ 10 \\ 8 \end{bmatrix}$

$R_2^\top = R_2$ (symmetric matrix) {\color{red}\checkmark}

$-R_2^\top\mathbf{d}_2 = -\begin{bmatrix} 1 & 0 & 0 \\ 0 & -1 & 0 \\ 0 & 0 & -1 \end{bmatrix}\begin{bmatrix} -5 \\ 10 \\ 8 \end{bmatrix} = \begin{bmatrix} 5 \\ 10 \\ 8 \end{bmatrix}$ {\color{red}\checkmark}

$$H_2^{-1} = \begin{bmatrix} 1 & 0 & 0 & 5 \\ 0 & -1 & 0 & 10 \\ 0 & 0 & -1 & 8 \\ 0 & 0 & 0 & 1 \end{bmatrix}$$

\textbf{Verification:} $H_2 \cdot H_2^{-1} = I$ {\color{red}\checkmark}

\item \textbf{Calculate workpiece position relative to robot base} \hfill \textbf{[4 marks]}

$H_{workpiece}^{robot} = H_2^{-1} \cdot H_1$ {\color{red}\checkmark}

$$H_{workpiece}^{robot} = \begin{bmatrix} 1 & 0 & 0 & 5 \\ 0 & -1 & 0 & 10 \\ 0 & 0 & -1 & 8 \\ 0 & 0 & 0 & 1 \end{bmatrix} \begin{bmatrix} 0 & -1 & 0 & 3 \\ 1 & 0 & 0 & 5 \\ 0 & 0 & -1 & 4 \\ 0 & 0 & 0 & 1 \end{bmatrix}$$ {\color{red}\checkmark}

$$= \begin{bmatrix} 0 & -1 & 0 & 8 \\ -1 & 0 & 0 & -5 \\ 0 & 0 & 1 & -4 \\ 0 & 0 & 0 & 1 \end{bmatrix}$$ {\color{red}\checkmark}

\textbf{Position:} $(8, -5, -4)$ relative to robot base {\color{red}\checkmark}

\item \textbf{Camera rotated $45^\circ$ about Y-axis} \hfill \textbf{[6 marks]}

Rotation matrix: $R_y(45^\circ) = \begin{bmatrix} 0.707 & 0 & 0.707 \\ 0 & 1 & 0 \\ -0.707 & 0 & 0.707 \end{bmatrix}$ {\color{red}\checkmark}

New camera transform: $H_{camera}' = \begin{bmatrix} R_y(45^\circ) & 0 \\ 0 & 1 \end{bmatrix}$ {\color{red}\checkmark}

Updated workpiece observation: $H_1' = H_{camera}' \cdot H_1$ {\color{red}\checkmark}

$$H_1' = \begin{bmatrix} 0.707 & 0 & 0.707 & 0 \\ 0 & 1 & 0 & 0 \\ -0.707 & 0 & 0.707 & 0 \\ 0 & 0 & 0 & 1 \end{bmatrix} \begin{bmatrix} 0 & -1 & 0 & 3 \\ 1 & 0 & 0 & 5 \\ 0 & 0 & -1 & 4 \\ 0 & 0 & 0 & 1 \end{bmatrix}$$ {\color{red}\checkmark}

$$= \begin{bmatrix} 0 & -0.707 & -0.707 & 4.95 \\ 1 & 0 & 0 & 5 \\ 0 & -0.707 & 0.707 & -0.70 \\ 0 & 0 & 0 & 1 \end{bmatrix}$$ {\color{red}\checkmark}

New workpiece relative to robot: $H_{workpiece}^{robot\,new} = H_2^{-1} \cdot H_1'$

New position computed from matrix multiplication {\color{red}\checkmark}
\end{enumerate}

\item A point $P(3,1,2)$ undergoes rotation and translation.

\begin{enumerate}[label=(\roman*)]
\item \textbf{Form homogeneous transformation matrix} \hfill \textbf{[3 marks]}

Rotation $180^\circ$ about Z: $R_z(180^\circ) = \begin{bmatrix} -1 & 0 & 0 \\ 0 & -1 & 0 \\ 0 & 0 & 1 \end{bmatrix}$ {\color{red}\checkmark}

Translation: $\mathbf{d} = \begin{bmatrix} 0 \\ 2 \\ 1 \end{bmatrix}$ {\color{red}\checkmark}

$$H = \begin{bmatrix} -1 & 0 & 0 & 0 \\ 0 & -1 & 0 & 2 \\ 0 & 0 & 1 & 1 \\ 0 & 0 & 0 & 1 \end{bmatrix}$$ {\color{red}\checkmark}

\item \textbf{Calculate final coordinates} \hfill \textbf{[3 marks]}

$$\mathbf{P}' = H \begin{bmatrix} 3 \\ 1 \\ 2 \\ 1 \end{bmatrix} = \begin{bmatrix} -1 & 0 & 0 & 0 \\ 0 & -1 & 0 & 2 \\ 0 & 0 & 1 & 1 \\ 0 & 0 & 0 & 1 \end{bmatrix} \begin{bmatrix} 3 \\ 1 \\ 2 \\ 1 \end{bmatrix}$$ {\color{red}\checkmark}

$$= \begin{bmatrix} -3 \\ 1 \\ 3 \\ 1 \end{bmatrix}$$ {\color{red}\checkmark}

\textbf{Final coordinates:} $(-3, 1, 3)$ {\color{red}\checkmark}
\end{enumerate}
\end{enumerate}

\subsection*{Question 3: 3-DOF Robot Manipulator - Solution \hfill \textbf{[20 marks]}}

\begin{enumerate}[label=(\alph*)]
\item For each individual rotation:

\begin{enumerate}[label=(\roman*)]
\item \textbf{Derive rotation matrices} \hfill \textbf{[6 marks]}

\textbf{Z-axis ($45^\circ$):}
$$R_z(45^\circ) = \begin{bmatrix} \cos 45^\circ & -\sin 45^\circ & 0 \\ \sin 45^\circ & \cos 45^\circ & 0 \\ 0 & 0 & 1 \end{bmatrix} = \begin{bmatrix} 0.707 & -0.707 & 0 \\ 0.707 & 0.707 & 0 \\ 0 & 0 & 1 \end{bmatrix}$$ {\color{red}\checkmark}{\color{red}\checkmark}

\textbf{Y-axis ($60^\circ$):}
$$R_y(60^\circ) = \begin{bmatrix} \cos 60^\circ & 0 & \sin 60^\circ \\ 0 & 1 & 0 \\ -\sin 60^\circ & 0 & \cos 60^\circ \end{bmatrix} = \begin{bmatrix} 0.5 & 0 & 0.866 \\ 0 & 1 & 0 \\ -0.866 & 0 & 0.5 \end{bmatrix}$$ {\color{red}\checkmark}{\color{red}\checkmark}

\textbf{X-axis ($30^\circ$):}
$$R_x(30^\circ) = \begin{bmatrix} 1 & 0 & 0 \\ 0 & \cos 30^\circ & -\sin 30^\circ \\ 0 & \sin 30^\circ & \cos 30^\circ \end{bmatrix} = \begin{bmatrix} 1 & 0 & 0 \\ 0 & 0.866 & -0.5 \\ 0 & 0.5 & 0.866 \end{bmatrix}$$ {\color{red}\checkmark}{\color{red}\checkmark}

\item \textbf{Verify orthogonality} \hfill \textbf{[3 marks]}

For $R_z(45^\circ)$:
$$R_z^\top R_z = \begin{bmatrix} 0.707 & 0.707 & 0 \\ -0.707 & 0.707 & 0 \\ 0 & 0 & 1 \end{bmatrix} \begin{bmatrix} 0.707 & -0.707 & 0 \\ 0.707 & 0.707 & 0 \\ 0 & 0 & 1 \end{bmatrix} = I$$ {\color{red}\checkmark}

Similarly verified for $R_y$ and $R_x$ {\color{red}\checkmark}{\color{red}\checkmark}
\end{enumerate}

\item Determine composite transformation:

\begin{enumerate}[label=(\roman*)]
\item \textbf{Composite rotation matrix} \hfill \textbf{[4 marks]}

$R_{composite} = R_x(30^\circ) \cdot R_y(60^\circ) \cdot R_z(45^\circ)$ {\color{red}\checkmark}

$$R_{composite} = \begin{bmatrix} 0.354 & -0.354 & 0.866 \\ 0.927 & 0.134 & -0.5 \\ 0.134 & 0.927 & 0 \end{bmatrix}$$ {\color{red}\checkmark}{\color{red}\checkmark}{\color{red}\checkmark}

\item \textbf{Homogeneous transformation} \hfill \textbf{[2 marks]}

$$H = \begin{bmatrix} 0.354 & -0.354 & 0.866 & 2 \\ 0.927 & 0.134 & -0.5 & 3 \\ 0.134 & 0.927 & 0 & 1 \\ 0 & 0 & 0 & 1 \end{bmatrix}$$ {\color{red}\checkmark}{\color{red}\checkmark}
\end{enumerate}

\item Using sensor point $\mathbf{P}(1,1,0)^\top$:

\begin{enumerate}[label=(\roman*)]
\item \textbf{Final position} \hfill \textbf{[3 marks]}

$$\mathbf{P}' = H \begin{bmatrix} 1 \\ 1 \\ 0 \\ 1 \end{bmatrix} = \begin{bmatrix} 2.0 \\ 4.061 \\ 2.061 \\ 1 \end{bmatrix}$$ {\color{red}\checkmark}{\color{red}\checkmark}

\textbf{Final position:} $(2.0, 4.061, 2.061)$ {\color{red}\checkmark}

\item \textbf{Verify distance preservation} \hfill \textbf{[2 marks]}

Initial: $\|\mathbf{P}\| = \sqrt{1^2 + 1^2 + 0^2} = 1.414$ {\color{red}\checkmark}

After rotation only (before translation): Distance preserved at $\sqrt{2}$ {\color{red}\checkmark}
\end{enumerate}
\end{enumerate}

\newpage

\hrule height 0.1pt
\vspace{-14pt}
\subsection*{SECTION B: Attempt \fbox{Only One} Question (20 marks)}
\vspace{-5pt}
\hrule height 0.1pt
\vspace{10pt}

\subsection*{Question 4: Pipe Welding Robot Orientation - Solution \hfill \textbf{[20 marks]}}

\begin{enumerate}[label=(\alph*)]
\item Determine surface normal vectors:

\begin{enumerate}[label=(\roman*)]
\item Top ($\theta = 0^\circ$): $\mathbf{n} = (0, 1, 0)$ {\color{red}\checkmark}
\item Right ($\theta = 90^\circ$): $\mathbf{n} = (0, 0, 1)$ {\color{red}\checkmark}
\item Bottom ($\theta = 180^\circ$): $\mathbf{n} = (0, -1, 0)$ {\color{red}\checkmark}
\item Left ($\theta = 270^\circ$): $\mathbf{n} = (0, 0, -1)$ {\color{red}\checkmark}
\end{enumerate}

\item \textbf{Rotation matrix for top position} \hfill \textbf{[4 marks]}

Torch Z-axis must align with $(0, 1, 0)$. Requires $-90^\circ$ rotation about X-axis: {\color{red}\checkmark}

$$R_{top} = R_x(-90^\circ) = \begin{bmatrix} 1 & 0 & 0 \\ 0 & \cos(-90^\circ) & -\sin(-90^\circ) \\ 0 & \sin(-90^\circ) & \cos(-90^\circ) \end{bmatrix}$$ {\color{red}\checkmark}

$$= \begin{bmatrix} 1 & 0 & 0 \\ 0 & 0 & 1 \\ 0 & -1 & 0 \end{bmatrix}$$ {\color{red}\checkmark}

Verification: Third column $(0, 1, 0)^\top$ confirms alignment {\color{red}\checkmark}

\item \textbf{Bottom position rotation and verification} \hfill \textbf{[5 marks]}

Requires $+90^\circ$ rotation about X-axis: {\color{red}\checkmark}

$$R_{bottom} = R_x(90^\circ) = \begin{bmatrix} 1 & 0 & 0 \\ 0 & 0 & -1 \\ 0 & 1 & 0 \end{bmatrix}$$ {\color{red}\checkmark}

\textbf{Verify orthogonality:}
$$R_{bottom}^\top R_{bottom} = \begin{bmatrix} 1 & 0 & 0 \\ 0 & 0 & 1 \\ 0 & -1 & 0 \end{bmatrix} \begin{bmatrix} 1 & 0 & 0 \\ 0 & 0 & -1 \\ 0 & 1 & 0 \end{bmatrix}$$ {\color{red}\checkmark}

$$= \begin{bmatrix} 1 & 0 & 0 \\ 0 & 1 & 0 \\ 0 & 0 & 1 \end{bmatrix} = I$$ {\color{red}\checkmark}

Orthogonal matrix confirmed {\color{red}\checkmark}

\item Extract Euler angles (ZYX convention):

\begin{enumerate}[label=(\roman*)]
\item \textbf{Top position} \hfill \textbf{[2 marks]}

From $r_{31} = 0$: $\beta = 0^\circ$ (pitch)

Roll: $\alpha = -90^\circ$, Yaw: $\gamma = 0^\circ$ {\color{red}\checkmark}{\color{red}\checkmark}

\item \textbf{Bottom position} \hfill \textbf{[2 marks]}

From $r_{31} = 0$: $\beta = 0^\circ$ (pitch)

Roll: $\alpha = +90^\circ$, Yaw: $\gamma = 0^\circ$ {\color{red}\checkmark}{\color{red}\checkmark}
\end{enumerate}

\item \textbf{X-axis rotations and gimbal lock} \hfill \textbf{[3 marks]}

All positions involve only X-axis rotations because the pipe axis (X-axis) is perpendicular to the circular cross-section. {\color{red}\checkmark}

\textbf{Gimbal lock:} Does NOT occur at any position because pitch $\beta = 0^\circ \neq \pm 90^\circ$ {\color{red}\checkmark}

The first rotation column $[1, 0, 0]^\top$ remains constant, confirming pure X-axis rotations. {\color{red}\checkmark}
\end{enumerate}

\subsection*{Question 5: Arbitrary Axis Rotation - Solution \hfill \textbf{[20 marks]}}

\begin{enumerate}[label=(\alph*)]
\item Normalize the rotation axis:

\begin{enumerate}[label=(\roman*)]
\item \textbf{Calculate magnitude} \hfill \textbf{[1 mark]}

$$\|\mathbf{V}\| = \sqrt{2^2 + 2^2 + 2^2} = \sqrt{12} = 3.464$$ {\color{red}\checkmark}

\item \textbf{Unit vector components} \hfill \textbf{[2 marks]}

$$V_x = V_y = V_z = \frac{2}{3.464} = 0.577$$ {\color{red}\checkmark}{\color{red}\checkmark}
\end{enumerate}

\item \textbf{Construct skew-symmetric matrix} \hfill \textbf{[3 marks]}

$$W = \begin{bmatrix} 0 & -V_z & V_y \\ V_z & 0 & -V_x \\ -V_y & V_x & 0 \end{bmatrix}$$ {\color{red}\checkmark}

$$= \begin{bmatrix} 0 & -0.577 & 0.577 \\ 0.577 & 0 & -0.577 \\ -0.577 & 0.577 & 0 \end{bmatrix}$$ {\color{red}\checkmark}{\color{red}\checkmark}

\item \textbf{Calculate $W^2$} \hfill \textbf{[4 marks]}

$$W^2 = W \times W = \begin{bmatrix} 0 & -0.577 & 0.577 \\ 0.577 & 0 & -0.577 \\ -0.577 & 0.577 & 0 \end{bmatrix} \begin{bmatrix} 0 & -0.577 & 0.577 \\ 0.577 & 0 & -0.577 \\ -0.577 & 0.577 & 0 \end{bmatrix}$$ {\color{red}\checkmark}

$$= \begin{bmatrix} -0.667 & 0.333 & 0.333 \\ 0.333 & -0.667 & 0.333 \\ 0.333 & 0.333 & -0.667 \end{bmatrix}$$ {\color{red}\checkmark}{\color{red}\checkmark}{\color{red}\checkmark}

\item Apply Rodrigues' formula:

\begin{enumerate}[label=(\roman*)]
\item \textbf{Evaluate trigonometric terms} \hfill \textbf{[1 mark]}

$\sin(90^\circ) = 1$, $(1-\cos(90^\circ)) = 1$ {\color{red}\checkmark}

\item \textbf{Compute rotation matrix} \hfill \textbf{[3 marks]}

$$R = I + W + W^2$$ {\color{red}\checkmark}

$$= \begin{bmatrix} 1 & 0 & 0 \\ 0 & 1 & 0 \\ 0 & 0 & 1 \end{bmatrix} + \begin{bmatrix} 0 & -0.577 & 0.577 \\ 0.577 & 0 & -0.577 \\ -0.577 & 0.577 & 0 \end{bmatrix} + \begin{bmatrix} -0.667 & 0.333 & 0.333 \\ 0.333 & -0.667 & 0.333 \\ 0.333 & 0.333 & -0.667 \end{bmatrix}$$ {\color{red}\checkmark}

$$= \begin{bmatrix} 0.333 & -0.244 & 0.911 \\ 0.911 & 0.333 & -0.244 \\ -0.244 & 0.911 & 0.333 \end{bmatrix}$$ {\color{red}\checkmark}
\end{enumerate}

\item \textbf{Transform point $P(1,0,0)^\top$} \hfill \textbf{[3 marks]}

$$\mathbf{P}' = R \begin{bmatrix} 1 \\ 0 \\ 0 \end{bmatrix} = \begin{bmatrix} 0.333 & -0.244 & 0.911 \\ 0.911 & 0.333 & -0.244 \\ -0.244 & 0.911 & 0.333 \end{bmatrix} \begin{bmatrix} 1 \\ 0 \\ 0 \end{bmatrix}$$ {\color{red}\checkmark}

$$= \begin{bmatrix} 0.333 \\ 0.911 \\ -0.244 \end{bmatrix}$$ {\color{red}\checkmark}

\textbf{Final position:} $(0.333, 0.911, -0.244)$ {\color{red}\checkmark}

\item \textbf{Verify orthogonality} \hfill \textbf{[3 marks]}

$$\det(R) = 0.333(0.333 \times 0.333 - (-0.244)(-0.244)) - (-0.244)(\ldots) + 0.911(\ldots)$$ {\color{red}\checkmark}

Computing the determinant: $\det(R) = 1.0$ {\color{red}\checkmark}

Since $\det(R) = +1$, the matrix represents a proper rotation (orthogonal and right-handed) {\color{red}\checkmark}
\end{enumerate}

\vfill
\begin{flushright}
\textit{End of Solutions}
\end{flushright}

\end{document}
